
\documentclass[smallcondensed]{svjour3}    % onecolumn (standard format)

%\documentclass[smallextended]{svjour3}     % onecolumn (second format)

%\documentclass[twocolumn]{svjour3}         % twocolumn

%
\smartqed  % flush right qed marks, e.g. at end of proof


%
% insert here the call for the packages your document requires

%
\usepackage{graphicx}
%
% \usepackage{mathptmx}      

%\usepackage{latexsym}
% etc.

%%%%%%%%%%%     IMPORTANT
%
% please place your own definitions here and don't use \def but
% \newcommand{}{}

%%%%%%%%%% 56


\begin{document}


\title{Insert your title here}
\subtitle{Do you have a subtitle? \\ If so, write it here}


%\titlerunning{Short form of title} % if too long for running head


\author{First Author         \and
        Second Author %etc.
}


%\authorrunning{Short form of author list} % if too long for running head


\institute{F. Author \at
              first address \\
              Tel.: +123-45-678910\\
              Fax: +123-45-678910\\
              \email{fauthor@example.com} }        
\maketitle


\begin{abstract}
Insert your abstract here. 
Include keywords, PACS and mathematical
subject classification numbers as needed.

\keywords{First keyword \and Second keyword \and More}
% \PACS{PACS code1 \and PACS code2 \and more}

\end{abstract}


\section{Introduction}
\label{intro}

Your text comes here. 
\section{How to introduce references in text}
The determination of the age of the family has been the concern of some authors who studied stable chaos (Firstauthor 1979, Author and Coauthor 1999, Another 2009).
Herein, we use the formulations given by Onemore (1979). That formulation 
was developed to study it specifically � by Secondmore (1989, 1993).
%\begin{acknowledgements}
%If you'd like to thank anyone, place your comments here
%and remove the percent signs.

%\end{acknowledgements}


% Non-BibTeX users please use

\begin{thebibliography}{}
%
% and use \bibitem to create references. Consult the Instructions

% for authors for reference list style.
%

%%%%%%%%%%% SEE EXAMPLES

\bibitem{label1}Firstname, X.:
Measure of the exponential
Celest. Mech. Dyn. Astron. {\bf 204}, 391-404 (1979)
\bibitem{label2}Author, X. and Coauthor, Y.: 
On remarks and question. In: J.Doe (ed) System with Degrees of Freedom, pp. 5668, Kluver, Dordrecht (1999)
\bibitem{label3}Another, X. .: 
Regular Dynamics, Springer, New York (2009)
\end{thebibliography}

\end{document}