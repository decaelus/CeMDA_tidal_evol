
\documentclass[smallcondensed]{svjour3}    % onecolumn (standard format)

%\documentclass[smallextended]{svjour3}     % onecolumn (second format)

%\documentclass[twocolumn]{svjour3}         % twocolumn

%
\smartqed  % flush right qed marks, e.g. at end of proof


%
% insert here the call for the packages your document requires

%
\usepackage{graphicx}
%
% \usepackage{mathptmx}      

%\usepackage{latexsym}
% etc.

%%%%%%%%%%%     IMPORTANT
%
% please place your own definitions here and don't use \def but
% \newcommand{}{}

\newcommand{\NumPlanetsConfirmed}{195}
\newcommand{\NumPlanetsMinMassOK}{148}
\newcommand{\NumPlanetsMinMassNotOK}{47}
\newcommand{\NumKeplerPlanetsPleTen}{1652}
\newcommand{\NumKeplerPlanetsPleOne}{111}
\newcommand{\NumSmallKeplerPlanetsPleOne}{85}
\newcommand{\NumLargePlanetsPleTen}{23}
\newcommand{\binsize}{10-min}
\newcommand{\mediumprecision}{600 ppm}
\newcommand{\newprecision}{80 ppm}
\newcommand{\boxcarwidth}{0.5 days}
\newcommand{\numbins}{50}

\newcommand{\tess}{\emph{TESS}}
\newcommand{\kepler}{\emph{Kepler}}
\newcommand{\corot}{\emph{CoRoT}}
\newcommand{\ktwo}{\emph{K2}}

%%%%%%%%%% 56


\begin{document}


\title{Tidal Decay and Disruption of Short-Period Gaseous Exoplanets}
%\subtitle{Do you have a subtitle? \\ If so, write it here}


%\titlerunning{Short form of title} % if too long for running head


\author{Brian Jackson         \and
        Emily Jensen %etc.
}


%\authorrunning{Short form of author list} % if too long for running head


\institute{Brian Jackson \at
		Boise State University, Dept.\ of Physics\\              
		1910 University Drive, Boise ID 83725 USA \\
		Tel.: 208-426-3723\\
              \email{bjackson@boisestate.edu} }        
\maketitle


\begin{abstract}
Many gaseous exoplanets in short-period orbits are on the verge or are
actually in the process of tidal disruption. Moreover, orbital
stability analysis shows tides can drive most known hot Jupiters to
spiral inexorably into their host stars. Thus, the coupled processes
of orbital decay and tidal disruption likely shape the observed
distribution of close-in exoplanets and may even be responsible for
producing the shortest-period rocky planets. However, the exact
outcome for a disrupting planet depends on its internal response to
mass loss and variable stellar insolation, and the accompanying
orbital evolution can act to enhance or inhibit the disruption
process, depending on the geometry of the atmospheric outflow. In some
cases, strong stellar insolation can produce a deep radiative zone in
a planet's atmosphere, which can also influence the disruption and
therefore the orbital evolution. Understanding these coupled processes
and making accurate predictions requires a model that includes both
the internal and the orbital evolution of the planet. In this
presentation, we will discuss our preliminary work on tidal decay and
disruption of close-in gas giants using the fully-featured and robust
Modules for Experiments in Stellar Astrophysics (MESA) suite, the
capabilities of which were recently upgraded to model gaseous planets
with inert, rocky cores.
\keywords{First keyword \and Second keyword \and More}
% \PACS{PACS code1 \and PACS code2 \and more}

\end{abstract}


\section{Introduction}
\label{sec:introduction}

\section{Introduction}
From wispy gas giants on the verge of tidal disruption to tiny rocky bodies already falling apart, extrasolar (or exo-) planets with orbital periods of several days and less challenge theories of planet formation and evolution. Since discovery of the first exoplanets, the population of known short period planets has grown dramatically, and they seem to be relatively common. Considering orbital periods $P < 50$ days, \cite{2010Sci...330..653H} estimated that 23\% of Sun-like FGK stars host planets with masses $M_\mathrm{p}$ between about 0.5 and 2 Earth masses $M_\mathrm{Earth}$, and, for $P < 12$ days, the authors estimated an occurrence rate of 1.2\% for planets with Jupiter-like masses. In addition, nearly half of the planets and planetary candidates reported by the \kepler~mission, \NumKeplerPlanetsPleTen~in total,  have $P < 10$ days \cite{2013ApJS..204...24B}. The results for the large-scale radial-velocity (RV) survey in \cite{2011arXiv1109.2497M} indicated that nearly half of solar-type stars host planets with $M_\mathrm{p} < 30\ M_\mathrm{Earth}$ and $P < 50$ days. These results establish that close-in planets represent a robust outcome of the evolution of planetary systems. Given current observational biases, short period planets dominate the available constraints on planetary composition, meteorology, etc., and so understanding their unique origins and natures is crucial for extrapolating those constraints to planets more distant from their host stars. 

Recent searches have found small, likely rocky planets with orbits reaching almost down to their host stars' surfaces \cite{2013A&A...555A..58O, 2013MNRAS.429.2001H, 2013ApJ...773L..15R, 2013ApJ...774...54S, 2013ApJ...779..165J}, and some short period candidates seem on the verge of tidal disruption or may be actively disrupting \cite{2012ApJ...752....1R, 2013arXiv1312.2054R}. The ratio of their orbital distances $a$ to the Roche limits $a_\mathrm{Roche}$ illustrates how close these planets and candidates are to disruption. Using the standard Roche limit expression for a completely fluid body $a_\mathrm{Roche} = 2.44 R_\mathrm{s} \left(\rho_\mathrm{p}/\rho_\mathrm{s}\right)^{1/3}$ (e.g., \cite{1959cbs..book.....K}), this ratio is
\begin{equation} 
a/a_\mathrm{Roche} = 0.41 \left( \frac{G P^2 \rho_\mathrm{p}}{3\pi} \right)^{1/3},
\label{eqn:aRoche_period}
\end{equation}
where $G$ is Newton's gravitational constant and $\rho_\mathrm{p}$ the planetary density. Figure \ref{fig:a-aRoche} shows this ratio, assuming $\rho_\mathrm{p} = 5.52$ g/cm$^3$ (Earth's density) where it is unknown, for all of \kepler's \NumSmallKeplerPlanetsPleOne~planets and candidates with $P <$ 1 day and $R_\mathrm{p} \le 1.6$ Earth radii ($R_\mathrm{Earth}$), with data taken from NASA's Exoplanet Archive\footnote{The observational parameters were taken from http://exoplanetarchive.ipac.caltech.edu/cgi-bin/ExoTables/nph-exotbls?dataset=cumulative, and all the data used here are available at http://www.astrojack.com/research.} and from \cite{2013ApJ...779..165J}. Table 1 provides the data used throughout this article. I will refer to these objects as very short period rocky planets or candidates. The boundary at 1-day is taken partly for convenience, and the considerations discussed here may also apply to planets with longer periods, depending on the conditions in those systems. As to the boundary in $R_\mathrm{p}$, \cite{Rogers_2014} showed most \kepler~candidates with $R_\mathrm{p} > 1.6\ R_\mathrm{Earth}$ likely have significant gaseous atmospheres and are not entirely rocky. \cite{Rogers_2014} does not simply give an $R_\mathrm{p}$ that divides rocky from gas/ice-rich planets, but I discuss primarily qualitative results here, for which this simple representation suffices. I focus here on very short period rocky planets, but many of the same considerations apply to gas-rich planets.

%\begin{figure}
%\includegraphics[width=\textwidth]{a_aRoche}
%\caption{Ratio of semi-major axes $a$ to Roche limits $a_\mathrm{Roche}$ for \kepler~confirmed planets (red squares) and KOI candidates (blue circles) with periods $P < 1$ day, plotted against their radii $R_\mathrm{p}$ in Earth radii. Some points are labeled with their Kepler (K-x) or KOI number. For candidates with unreported densities, I've assumed an Earth-like density of 5.52 g/cm$^3$ to calculate $a/a_\mathrm{Roche}$.}
%\label{fig:a-aRoche}
%\end{figure}

Figure \ref{fig:a-aRoche} includes two particularly noteworthy objects, \kepler~Object of Interest KOI-1843.03 and Kepler-78 b. Discovered by \cite{2013A&A...555A..58O}, KOI-1843.03 is the shortest period transiting candidate known, with $P =$ 4.25 hours, and orbits a star with a mass $M_\mathrm{s} = 0.509$ solar masses $M_\mathrm{Sun}$ reported by Exoplanet Archive. With Earth's density, $a/a_\mathrm{Roche} = 0.8$, suggesting KOI-1843.03 must be denser not to be disrupted. \cite{2013ApJ...773L..15R} argued KOI-1843.03's $\rho_\mathrm{p} \ge 7$ g/cm$^3$, using a modified Roche limit model, while the standard expression gives $\rho_\mathrm{p} \ge 9$ g/cm$^3$. Although KOI-1843.03 is unconfirmed, there are two other transiting candidates in the system, with $P \approx$ 4.2 and 6.4 days \cite{2013ApJS..204...24B}. According to \cite{2012ApJ...750..112L}, three false positives for one \kepler~target is exceedingly unlikely, so the KOI-1843 candidates are probably planets. Kepler-78 b is the shortest period transiting planet confirmed, orbiting a star with $M_\mathrm{s} = 0.83\ M_\mathrm{Sun}$, and RV follow-up gave $M_\mathrm{p} = 1.69$ Earth masses ($M_\mathrm{Earth}$) and $\rho_\mathrm{p} = 5.3$ g/cm$^3$ \cite{2013Natur.503..381H}. As part of a coordinated effort, \cite{2013Natur.503..377P} reported similar results. Although it has a longer orbital period than KOI-1843.03, Kepler-78 b is still close to disruption and, depending on its origin, may have experienced some disruption. The overall false positive rate for very short period rocky candidates is unknown but may figure into considerations of their origins (Section \ref{sec:frequency_vspp}).

In this article, I speculate on the origins and natures of \kepler's very short period likely rocky planets and planetary candidates. For brevity, I will often refer to them all, whether confirmed or not, as planets. Since they are some of the most dramatic examples, I focus on KOI-1843.03 and Kepler-78 b as prototypes. In Section \ref{sec:theoretical_considerations}, I discuss theoretical considerations regarding the planets' origins and natures. In Section \ref{sec:observational_predictions}, I discuss possible observations and (potentially) testable predictions. I conclude in Section \ref{sec:conclusions}. In Appendix \ref{app:estimating_masses}, I discuss methods used for estimating the unknown masses for candidate planets.

All of the origin scenarios I suggest here have been considered in previous studies. For example, \cite{2008MNRAS.384..663R} considered scenarios similar to those I describe here (and others I do not) and provided some specific observational predictions for each. However, that study was published before small planets around Sun-like stars with $P < 1$ day were discovered (e.g., CoRoT-7 b was announced in 2009 -- \cite{2009A&A...506..287L}), and the proximities of these planets to their host stars more strongly circumscribe the possible origins than for longer-period (but still close-in) rocky planets. Thus, this study represents an update to those earlier studies in light of these new discoveries.
%\begin{acknowledgements}
%If you'd like to thank anyone, place your comments here
%and remove the percent signs.

%\end{acknowledgements}

\bibliography{Jackson_tidal-decay-disrupt}

\end{document}