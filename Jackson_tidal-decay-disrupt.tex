
\documentclass[smallcondensed]{svjour3}    % onecolumn (standard format)

%\documentclass[smallextended]{svjour3}     % onecolumn (second format)

%\documentclass[twocolumn]{svjour3}         % twocolumn

%
\smartqed  % flush right qed marks, e.g. at end of proof


%
% insert here the call for the packages your document requires

%
\usepackage{graphicx}
%
% \usepackage{mathptmx}      

%\usepackage{latexsym}
% etc.

%%%%%%%%%%%     IMPORTANT
%
% please place your own definitions here and don't use \def but
% \newcommand{}{}

%%%%%%%%%% 56


\begin{document}


\title{Tidal Decay and Disruption of Short-Period Gaseous Exoplanets}
%\subtitle{Do you have a subtitle? \\ If so, write it here}


%\titlerunning{Short form of title} % if too long for running head


\author{Brian Jackson         \and
        Emily Jensen %etc.
}


%\authorrunning{Short form of author list} % if too long for running head


\institute{Brian Jackson \at
		Boise State University, Dept.\ of Physics\\              
		1910 University Drive, Boise ID 83725 USA \\
		Tel.: 208-426-3723\\
              \email{bjackson@boisestate.edu} }        
\maketitle


\begin{abstract}
Many gaseous exoplanets in short-period orbits are on the verge or are
actually in the process of tidal disruption. Moreover, orbital
stability analysis shows tides can drive most known hot Jupiters to
spiral inexorably into their host stars. Thus, the coupled processes
of orbital decay and tidal disruption likely shape the observed
distribution of close-in exoplanets and may even be responsible for
producing the shortest-period rocky planets. However, the exact
outcome for a disrupting planet depends on its internal response to
mass loss and variable stellar insolation, and the accompanying
orbital evolution can act to enhance or inhibit the disruption
process, depending on the geometry of the atmospheric outflow. In some
cases, strong stellar insolation can produce a deep radiative zone in
a planet's atmosphere, which can also influence the disruption and
therefore the orbital evolution. Understanding these coupled processes
and making accurate predictions requires a model that includes both
the internal and the orbital evolution of the planet. In this
presentation, we will discuss our preliminary work on tidal decay and
disruption of close-in gas giants using the fully-featured and robust
Modules for Experiments in Stellar Astrophysics (MESA) suite, the
capabilities of which were recently upgraded to model gaseous planets
with inert, rocky cores.
\keywords{First keyword \and Second keyword \and More}
% \PACS{PACS code1 \and PACS code2 \and more}

\end{abstract}


\section{Introduction}
\label{intro}

Your text comes here. 
\section{How to introduce references in text}
The determination of the age of the family has been the concern of some authors who studied stable chaos (Firstauthor 1979, Author and Coauthor 1999, Another 2009).
Herein, we use the formulations given by Onemore (1979). That formulation 
was developed to study it specifically � by Secondmore (1989, 1993).
%\begin{acknowledgements}
%If you'd like to thank anyone, place your comments here
%and remove the percent signs.

%\end{acknowledgements}


% Non-BibTeX users please use

\begin{thebibliography}{}
%
% and use \bibitem to create references. Consult the Instructions

% for authors for reference list style.
%

%%%%%%%%%%% SEE EXAMPLES

\bibitem{label1}Firstname, X.:
Measure of the exponential
Celest. Mech. Dyn. Astron. {\bf 204}, 391-404 (1979)
\bibitem{label2}Author, X. and Coauthor, Y.: 
On remarks and question. In: J.Doe (ed) System with Degrees of Freedom, pp. 5668, Kluver, Dordrecht (1999)
\bibitem{label3}Another, X. .: 
Regular Dynamics, Springer, New York (2009)
\end{thebibliography}

\end{document}