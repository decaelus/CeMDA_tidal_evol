% LaTeX-template for ExoPlanetNews abstract. v1.1 (March 2013)
% Please see the comments below. All lines beginning "%" are not processed or printed
%
%------------------------------------------------------------------------------------------------------------
%
% Remove the "%" from the following 3 lines to preview; please comment out these lines again prior to submission
\documentclass{article}
\usepackage{ep}
\begin{document}
%
%------------------------------------------------------------------------------------------------------------
%
% Title of your paper and short author list (surnames only, no initials)
% Use: 'Author1 et al.' if more than three authors
\title{Tidal Decay and Stable Roche-Lobe Overflow of Short-Period Gaseous Exoplanets}{Brian Jackson et al.}
%
% or for CONFERENCES use the style:
% \title{Conference Title}{Physical location of Conference}
%
% or for JOB ADVERTS use the style:
% \title{Post Title}{Institute at which post to be held}
%
% or for MISCELLANEOUS ANNOUNCEMENTS use the style:
% \title{Announcement Title}{Institute/Location to which announcement applies}
%
%------------------------------------------------------------------------------------------------------------
%
% Author(s) of your paper (initials then surnames)
% Use \inst{1} etc. for numbering of different institutes.
%
\author{Brian Jackson\inst{1}, Emily Jensen\inst{1}, Sarah Peacock\inst{2}, Phil Arras\inst{3}, Kaloyan Penev\inst{4}}
%
% or for CONFERENCES use the style:
% \author(Conference Chair(s)}
%
% or for JOB ADVERTS use the style:
% \author{Person offering the job}
%
% or for MISCELLANEOUS ANNOUNCEMENTS use the style:
% \author{Person making the announcement}
%
% Institutes. In the first argument
% Give the numbers as used in the \author command above
%
\instlist{1}{Boise State University, Dept. of Physics, 1910 University Drive, Boise ID 83725 USA}
\instlist{2}{Lunar and Planetary Laboratory, University of Arizona, 1629 E University Blvd, Tucson, AZ 85721-0092}
\instlist{3}{Department of Astronomy, University of Virginia, Charlottesville, VA 22904-4325, USA}
\instlist{4}{Department of Astrophysical Sciences, Princeton University, NJ 08544, USA}
%
% Alternately, if all authors are from the SAME single institute, use:
%\institute{My University, Down the Road, Any Town, UK}
%
%------------------------------------------------------------------------------------------------------------
%
% Status of your paper. 
%First Argument: The journal where it will appear.
% Second Argument. The state - should be one of: "in press" or "published".
%
% If it is already published, please provide us with the ADS-Bibcode,
% otherwise give the arXiv preprint code in the style: "arXiv:nnnn.mmmm"
%
\status{Celestial Mechanics and Dynamical Astronomy}{in press (http://adsabs.harvard.edu/abs/2016arXiv160300392J/arXiv:1603.00392)}
%
% or for CONFERENCES use the style:
% \status{Location of Conference}{Dates of Conference}
%
% or for JOB ADVERTS use the style:
% \status{Location of Post}{Job start date}
%
% or for MISCELLANEOUS ANNOUNCEMENTS use the style:
% \status{Location of Announcement}{Relevant date for announcement}
%
% You should use the following abbreviations:
% \mnras   Monthly Notices of the Royal Astronomical Society
% \aj      Astronomical Journal
% \apj     Astrophysical Journal
% \apjl    Astrophysical Journal Letters
% \aa      Astronomy \& Astrophysics
% \aal     Astronomy \& Astrophysics Letters
% \pasp    Publications of the Astronomical Society of the Pacific
% \aas     American Astronomical Society Meeting
% \pasj    Publications of the Astronomical Society of Japan
%
%
%-----------------------------------------------------------------------------------------------------------
%
% The abstract body
%
\abstract{
Many gaseous exoplanets in short-period orbits are on the verge or are in the process of Roche-lobe overflow (RLO). Moreover, orbital stability analysis shows tides can drive many hot Jupiters to spiral inevitably toward their host stars. Thus, the coupled processes of orbital evolution and RLO likely shape the observed distribution of close-in exoplanets and may even be responsible for producing some of the short-period rocky planets. However, the exact outcome for an overflowing planet depends on its internal response to mass loss, and the accompanying orbital evolution can act to enhance or inhibit RLO. In this study, we apply the fully-featured and robust Modules for Experiments in Stellar Astrophysics (MESA) suite to model RLO of short-period gaseous planets. We show that, although the detailed evolution may depend on several properties of the planetary system, it is largely determined by the core mass of the overflowing gas giant. In particular, we find that the orbital expansion that accompanies RLO often stops and reverses at a specific maximum period that depends on the core mass. We suggest that RLO may often strand the remnant of a gas giant near this orbital period, which provides an observational prediction that can corroborate the hypothesis that short-period gas giants undergo RLO. We conduct a preliminary comparison of this prediction to the observed population of small, short-period planets and find some planets in orbits that may be consistent with this picture. To the extent that we can establish some short-period planets are indeed the remnants of gas giants, that population can elucidate the properties of gas giant cores, the properties of which remain largely unconstrained.
}
%
% Optional Figure caption: (just paste in here - it will be typeset separately)
%\caption{Mass (red lines) and orbital (blue lines) evolution for hot Jupiter systems with initial periods $P_{0} =$ 3 days and $Q_\star = 10^5$ amd a variety of initial envelope masses $M_{\rm env,\ 0}$ and core masses $M_{\rm core}$. The different linestyles indicate different planetary parameters. (a) Hot Jupiters with $M_{\rm env,\ 0} =$ 0.3, 1, and 3 ${\rm M_{Jup}}$ and $M_{\rm core}$ fixed at 10 ${\rm M_{Earth}}$. (b) Hot Jupiters with $M_{\rm env,\ 0} = 1\ {\rm M_{Jup}}$ and $M_{\rm core} = $ 1, 5, 10, and 30 ${\rm M_{Earth}}$. These calculations assume $\delta \gamma = 0$, i.e. the orbital angular momentum is completely conserved.}
%------------------------------------------------------------------------------------------------------------
%
% Download: Website with the preprint and/or additional information/data (CHANGE this from the current value)
%
\download{http://arxiv.org/abs/1603.00392}
%
% Contact: E-Mail of the responsible person (CHANGE this from the current value)
%
\contact{bjackson@boisestate.edu}
%
%------------------------------------------------------------------------------------------------------------
%
% Remove the "%" from the next line to preview; comment it out again before submission
%
\end{document}
%
%------------------------------------------------------------------------------------------------------------
